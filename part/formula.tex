\section{强大的数学公式}
\begin{frame}[fragile]{写在前面}
	除了一些最简单的数学公式,大部分都需要宏包支持。建议平时需要输入公式时就直接使用以下四个宏包
\begin{lstlisting}
\usepackage{amsmath,amsfonts,amssymb,mathtools}
\end{lstlisting}
	这能够支持大多数的数学公式,而不必去纠结什么时候应该用哪一个、哪一个是多余的。

	\vspace{2ex}
	在讨论更多的数学环境前,我们统一使用 \cprotect\fbox{\verb|$|$\dots$\verb|$|} 输入行内(inline)公式,\cprotect\fbox{\verb|\[|$\dots$\verb|\]|} 输入行间(displayed)公式。

	\vspace{2ex}
	\textcolor{red}{注意}:凡是要用到数学字符的地方必须进入数学环境,有些字符虽然不在数学环境下也能输入,但是区别很大!
	\begin{table}
		\begin{tabular}{rlrl}
			\textrm{1+1=2} & $1+1=2$ \quad & \quad \textrm{a sin x} & $a \sin x$
		\end{tabular}
	\end{table}
\end{frame}

\subsection{数学符号}
\begin{frame}[fragile]{字母表与数学字体}
	在公式环境内默认使用 \cprotect\fbox{\verb|\mathnormal|} 字体,与正文环境下默认字体对比如下
	\begin{itemize}
		\item Capital
		\begin{itemize}
			\item \textrm{ABCDEFGHIJKLMNOPQRSTUVWXYZ}
			\item $ABCDEFGHIJKLMNOPQRSTUVWXYZ$
		\end{itemize}
		\item Lower case
		\begin{itemize}
			\item \textrm{abcdefghijklmnopqrstuvwxyz}
			\item $abcdefghijklmnopqrstuvwxyz$
		\end{itemize}
	\end{itemize}
	其明显特征是斜体且间距较正文大一些。

	\vspace{2ex}
	而\cprotect\fbox{\verb|\mathit|}, \cprotect\fbox{\verb|\mathrm|}, \cprotect\fbox{\verb|\mathbf|}, \cprotect\fbox{\verb|\mathsf|}, \cprotect\fbox{\verb|\mathtt|} 这些数学字体通常直接使用相对应的正文字体\cprotect\fbox{\verb|\text**|}。
\end{frame}
\begin{frame}[fragile]{选用正确的字体}
	\begin{itemize}
		\item 变量使用默认字体,如$x, y, z$
		\item $\mathrm{e}, \mathrm{i}$等常量应使用罗马体 \cprotect\fbox{\verb|\mathrm|}
		\item 集合$\mathbb{N}, \mathbb{R}$等可使用 \verb|amssymb|提供的黑板粗体 \cprotect\fbox{\verb|\mathbb|}
		\item 向量$\bm{v}$应加粗,实现方法有很多,这里推荐 \verb|bm| 宏包的 \cprotect\fbox{\verb|\bm|} 命令
	\end{itemize}
\end{frame}
\begin{frame}[fragile]{希腊字母}
	\begin{itemize}
		\item 小写希腊字母
		\begin{itemize}
			\item 一般 \cprotect\fbox{\verb|\+小写拼写|},如 \cprotect\fbox{\verb|\alpha|} 即输出$\alpha$
			\item 注意$\mu, \nu$分别为 \cprotect\fbox{\verb|\mu|},\cprotect\fbox{\verb|\nu|}
			\item 一部分希腊字母有变体,这些变体在数学公式中常常用到,只需在前面加上 \verb|var|
				\begin{itemize}
				\item[] \small $\epsilon \rightarrow \varepsilon$ \qquad $\phi \rightarrow \varphi$ \qquad
					$\theta \rightarrow \vartheta$ \qquad $\sigma \rightarrow \varsigma$ \qquad
					$\pi \rightarrow \varpi$
				\end{itemize}
		\end{itemize}
		\item 大写希腊字母
		\begin{itemize}
			\item 一部分大写希腊字母与拉丁字母形状相同,直接使用即可,如$\mathrm{A, B}$
			\item 将小写希腊字母输入时首字母大写即可得到其大写形式,如 \cprotect\fbox{\verb|\Gamma|} 得到$\Gamma$
			\item 大写希腊字母一般用正体,前面加 \cprotect\fbox{\verb|var|} 或使用 \cprotect\fbox{\verb|\mathnormal|} 可得到倾斜形式
		\end{itemize}
		\item 小写希腊字母直立体形式?
		\begin{itemize}
			\item 数字字体宏包 \verb|upgreek|,在前面加上 \cprotect\fbox{\verb|up|}
			\item \cprotect\fbox{\verb|\pi|} $\rightarrow$ \cprotect\fbox{\verb|\uppi|} \qquad $\pi \rightarrow \uppi$
		\end{itemize}
	\end{itemize}
\end{frame}
\begin{frame}[fragile]{常用符号}
	\begin{table}\caption{数学普通符号(部分)}
		\begin{tabular}{llllll}
			\toprule
			$\hbar$ & \verb|\hbar| & $\ell$ & \verb|\ell| & $\partial$ & \verb|\partial| \\
			$\infty$ & \verb|\infty| & $\prime$ & \verb|\prime| & $\emptyset$ & \verb|\emptyset| \\
			$\nabla$ & \verb|\nabla| & $\forall$ & \verb|\forall| & $\exists$ & \verb|exists| \\
			$\neg$ & \verb|\neg| & $\varnothing$ & \verb|\varnothing| & $\angle$ & \verb|\angle| \\
			$\clubsuit$ & \verb|\clubsuit| & $\diamondsuit$ & \verb|\diamondsuit| & $\heartsuit$ & \verb|\heartsuit| \\
			$\spadesuit$ & \verb|\spadesuit| & $\triangle$ & \verb|\triangle| & $\square$ & \verb|\square| \\
			$\flat$ & \verb|\flat| & $\natural$ & \verb|\natural| & $\sharp$ & \verb|\sharp|\\
			\bottomrule
		\end{tabular}
	\end{table}
\end{frame}
\begin{frame}[fragile]{常用符号}
	\begin{table}\caption{可同时用在文本和数学模式中的符号}
		\begin{tabular}{llllllll}
			\toprule
			$\#$ & \verb|\#| & $\&$ & \verb|\&| & $\%$ & \verb|\%| & $\$$ & \verb|\$| \\
			$\_$ & \verb|\_| & $\{$ & \verb|\{| & $\}$ & \verb|\}| & & \\
			$\P$ & \verb|\P| & $\S$ & \verb|\S| & $\dag$ & \verb|\dag| & $\ddag$ & \verb|\ddag| \\
			$\copyright$ & \verb|\copyright| & $\pounds$ & \verb|\pounds| & $\ldots$ & \verb|\ldots| & & \\
			$\checkmark$ & \verb|\checkmark| & $\circledR$ & \verb|\circledR| & $\maltese$ & \verb|\maltese| & $\yen$ & \verb|\yen| \\
			\bottomrule
		\end{tabular}
	\end{table}
	\begin{table}\caption{希伯来字母}
		\begin{tabular}{llllllll}
			\toprule
			$\aleph$ & \verb|\aleph| & $\beth$ & \verb|\beth| & $\daleth$ & \verb|\daleth| & $\gimel$ & \verb|\gimel| \\
			\bottomrule
		\end{tabular}
	\end{table}
\end{frame}
\begin{frame}[fragile]{数学算子}{Math operator}
	\begin{itemize}
		\item 巨算符(large operator)
		\begin{itemize}
			\item 所谓巨算符是大小可变的运算符
			\item 其在行内与行间大小是不同的
			\item 一般可以添加有上下标(数学结构中会讲)
			\item 行内行间上下标的形式也有所不同
		\end{itemize}
	\end{itemize}
	\begin{align*}
		\int_a^b & \textstyle \int_a^b & \iint_m^n & \textstyle \iint_m^n & \oint_{|z|=1} & \textstyle \oint_{|z|=1} \\
		\sum_{n=0}^{\infty} & \textstyle \sum_{n=0}^{\infty} & \prod_{1 \le i < j \le n} & \textstyle \prod_{1 \le i < j \le n} & \bigcup\nolimits_{i=1}^{n} & \textstyle \bigcup\limits_{i=1}^{n}
	\end{align*}
	\small \quad \cprotect\fbox{\verb|\limits|} \quad \cprotect\fbox{\verb|\nolimits|} \quad 行内公式尽量避免上下限放在算符上下位置
\end{frame}
\begin{frame}[fragile]{数学算子}{Math operator}
	\begin{itemize}
		\item 文字名称的算子
		\begin{itemize}
			\item 使用直立罗马体排印,一般 \cprotect\fbox{\verb|\+名称|} 即可
			\item 分为两类,带上下限的与不带上下限的
			\begin{itemize}
				\item $\log \quad \sin \quad \cosh \quad \arctan \quad \exp$ 
				\item $\lim \quad \max \quad \min \quad \sup \quad \inf$ 
			\end{itemize}
			\item 带上下限的算子使用起来与巨算符相似\\
				\qquad $\lim\limits_{x \to 0} \qquad \lim_{x \to 0}$
			\item 如果需要定义新的算子,可以在导言区使用 \verb|\DeclareMathOperator(*)|,带 \verb|*|表示带上下限
			\begin{itemize}
				\item \verb|\DeclareMathOperator{\arccot}{arccot}| \\ $\arccot$
				\item \verb|\DeclareMathOperator{\sign}{sgn}| \\ $\sign$
				\item \verb|\DeclareMathOperator*{\limsup}{lim\,sup}| \\ $\limsup$
			\end{itemize}
		\end{itemize}
	\end{itemize}
\end{frame}
\begin{frame}[fragile]{二元运算符与关系符}
	\begin{itemize}
		\item $\cprotect\overset{\cprotect\fbox{\scriptsize\verb|+|}}{+}, 
	\cprotect\underset{\cprotect\fbox{\scriptsize\verb|-|}}{-}, 
	\cprotect\overset{\cprotect\fbox{\scriptsize\verb|\times|}}{\times}, 
	\cprotect\underset{\cprotect\fbox{\scriptsize\verb|\div|}}{\div}$等二元运算符与
	$\cprotect\overset{\cprotect\fbox{\scriptsize\verb|=|}}{=}, 
	\cprotect\underset{\cprotect\fbox{\scriptsize\verb|>|}}{>}, 
	\cprotect\overset{\cprotect\fbox{\scriptsize\verb|<|}}{<}, 
	\cprotect\underset{\cprotect\fbox{\scriptsize\verb|\ne|}}{\ne}, 
	\cprotect\overset{\cprotect\fbox{\scriptsize\verb|\le|}}{\le}$\textrm{,} $
	\cprotect\underset{\cprotect\fbox{\scriptsize\verb|\ge|}}{\ge}$\textrm{,} $
	\cprotect\overset{\cprotect\fbox{\scriptsize\verb|\ll|}}{\ll}$\textrm{,} $
	\cprotect\underset{\cprotect\fbox{\scriptsize\verb|\gg|}}{\gg}$等关系符与前后变量之间会保留一定的空隙,且行内公式可以在它们的后面断行。
	\item 二元关系符数量庞大,除了一些最常用的,一般有需要的时候随用随查即可。
	\item 使用 \cprotect\fbox{\verb|\mathbin|} 和 \cprotect\fbox{\verb|\mathrel|} 可以将其参数当作二元运算符和二元关系符来对待,可用于定义新运算或一些其他用途。
	\item 箭头符号与逻辑符号使用时同样可以查阅,不再赘述。
	\end{itemize}
\end{frame}
\begin{frame}[fragile]{定界符}
	\[\{\frac{\partial y_j}{\partial x_i} | i=1,2,\cdots,n,\, j=1,2,\cdots,m\}\]
	\[\left\{\frac{\partial y_j}{\partial x_i} \middle| i=1,2,\cdots,n,\, j=1,2,\cdots,m\right\}\]
	上例中的$\{\,|\,\}$均为定界符,其大小可以根据式子的大小而改变。
	\begin{itemize}
		\item 括号定界符
		\[(\,)\quad[\,]\quad\{\,\}\quad\langle\,\rangle\quad\lfloor\,\rfloor\quad\lceil\,\rceil\]
		\item 非括号定界符
		\[/\quad\backslash\quad|\quad\|\]
		\item 输入
		\begin{itemize}
			\item \verb|( ) [ ] \{ \} \langle \rangle|\\
				\verb|\lfloor \rfloor \lceil \rceil|
			\item \verb!/ \backslash | \|!
		\end{itemize}
	\end{itemize}
\end{frame}
\begin{frame}[fragile]{如何确定定界符大小}
	\begin{itemize}
		\item 自动调整
		\begin{itemize}
			\item 使用 \cprotect\fbox{\verb|\left( ... \right)|},左右必须成对出现
			\item 如果只需要一个定界符,可以用 \cprotect\fbox{\verb|.|} 表示空的定界符,如 \cprotect\fbox{\verb|\left.|}
			\item 需要三个定界符的话可使用 \cprotect\fbox{\verb|\middle|} 放在中间,如前例
		\end{itemize}
		\item 手动调整
		\begin{itemize}
			\item 提供了 \cprotect\fbox{\verb|\big|},\cprotect\fbox{\verb|\Big|}, \cprotect\fbox{\verb|\bigg|},\cprotect\fbox{\verb|\Bigg|} 四种大小供手动调节
			\item 可以在后面加上 \cprotect\fbox{\verb|l m r|} 配套使用
			\item 自动调整效果不理想时,可以试试手动调整
			\item 在明确“左右”(如 \cprotect\fbox{\verb|\left \right|} 或 \cprotect\fbox{\verb|\Bigl \Bigr|})时,可直接使用 \cprotect\fbox{\verb|<>|} 表示尖括号而不致与$<>$混淆
		\end{itemize}
	\end{itemize}
\end{frame}
\begin{frame}[fragile]{标点符号}
	\begin{itemize}
		\item 数学环境内应使用英文标点,否则会报错
		\item 常用的标点符号有$, \quad ; \quad ! \quad ? \quad \colon$
		\item 应注意的是\cprotect\fbox{\verb|:|}是作为二元关系符存在的(如$f(x):=x, \{x : x>0\}$),而作为标点的冒号是 \cprotect\fbox{\verb|\colon|},其两侧间距不同,表示比例时则最好用 \cprotect\fbox{\verb|\mathbin{:}|}
		\item 省略号有很多种,在矩阵中常用到,平时最常用的是 \cprotect\fbox{\verb|\cdots|} $\cdots$ 和 \cprotect\fbox{\verb|\ldots|} $\ldots$ (或直接使用 \cprotect\fbox{\verb|\dots|} 让latex自己选择)
		\item 标点后是无法换行的,像$f(x,y,z)$谁也不希望在中间断行,而$a_1$\textrm{,} $a_2$\textrm{,} $a_3$\textrm{,} $\cdots$\textrm{,} $a_n$希望能断行时应将标点置于数学环境之外并加上空格
		\item 数学环境中空格不起作用,但可以分隔各部分让公式结构更清晰
	\end{itemize}
\end{frame}

\subsection{数学结构}
\begin{frame}[fragile]{上下标}
	\begin{itemize}
		\item 上标用 \verb|^| 表示,下标用 \verb|_| 表示
		\item 超过一个字符时就应使用$\{\,\}$
		\item \verb|'| 是一种特殊的上标,相当于 \cprotect\fbox{\verb|^\prime|}
		\item 使用 \cprotect\fbox{\verb|\overset|} 和 \cprotect\fbox{\verb|\underset|}可以在任意符号上下方添加标记
	\end{itemize}
\end{frame}
\begin{frame}[fragile]{分式}
	\begin{itemize}
		\item 使用 \cprotect\fbox{\verb|\frac{num}{den}|} 输入分式,先输入分子(numerator),后输入分母(denominator)
		\item 行内默认使用正文格式(text style)分式,行间默认使用显示格式(display style)分式,可以使用 \cprotect\fbox{\verb|\tfrac|} 与 \cprotect\fbox{\verb|\dfrac|} 指定使用某一种
		\item 行内使用分式时,$a/b$效果往往比$\frac ab$更好
		\item 连分式用 \cprotect\fbox{\verb|\cfrac|},这里不细讲
		\item 顺带一提,\cprotect\fbox{\verb|\textstyle|} 和 \cprotect\fbox{\verb|\displaystyle|} 可以指定整条公式以哪种方式显示
	\end{itemize}
\end{frame}
\begin{frame}[fragile]{根式}
	\begin{itemize}
		\item \cprotect\fbox{\verb|\sqrt[root]{arg}|}
		\item \texttt{[\,]}内为可选项,用\cprotect\fbox{\verb|\uproot|} 与 \cprotect\fbox{\verb|\leftroot|} 可以微调\texttt{root}位置,参数为整数
		\item 被开方数不是整数或结构复杂时,通常改用等价的指数形式
		\item 使用 \cprotect\fbox{\verb|\vphantom|} 占位来获得统一的高度\\[1ex]
		\verb|\sqrt{\frac 12} < \sqrt{\vphantom{\frac12}2}|
		\[\sqrt{\frac 12} < \sqrt{\vphantom{\frac12}2}\]
		\item \cprotect\fbox{\verb|\mathstrut|} 用来平衡不同高度和深度的字母\\[1ex]
		\verb|\sqrt b \sqrt y|\\
		\verb|\sqrt{\mathstrut b} \sqrt{\mathstrut y}|
		\[\sqrt b \sqrt y \quad \rightarrow \quad \sqrt{\mathstrut b} \sqrt{\mathstrut y}\]
	\end{itemize}
\end{frame}
\begin{frame}[fragile]{矩阵}
	\begin{itemize}
		\item 输入矩阵结构需要矩阵环境,它们的区别在于外面的括号不同
		\begin{itemize}
			\item \texttt{ matrix}环境 \quad 无括号
			\item \texttt{pmatrix}环境 \quad 圆括号
			\item \texttt{bmatrix}环境 \quad 方括号
			\item \texttt{Bmatrix}环境 \quad 大括号
			\item \texttt{vmatrix}环境 \quad 单竖线
			\item \texttt{Vmatrix}环境 \quad 双数线
		\end{itemize}
		\item 以一个最简单的二阶矩阵为例\\
		\begin{minipage}[c]{0.5\textwidth}
\begin{lstlisting}
\[ \begin{pmatrix}
       a_{11} & a_{12} \\
       a_{21} & a_{22}
   \end{pmatrix} \]
\end{lstlisting}
		\end{minipage}
		\begin{minipage}[c]{0.4\textwidth}
			\[ \begin{pmatrix}
				a_{11} & a_{12} \\
				a_{21} & a_{22}
			\end{pmatrix} \]
		\end{minipage}
	\end{itemize}
\end{frame}
\subsection{数学环境}
\begin{frame}[fragile]{概述}
	我们先前使用的 \cprotect\fbox{\verb|\[|$\dots$\verb|\]|} 实质是
\begin{lstlisting}
\begin{equation*}
    ...
\end{equation*}
\end{lstlisting}
	的缩写,其中“$*$”表示不给该公式标序。

	\vspace{2ex}
	出于各方面的需求,我们并不满足于这样一个简单居中、无法换行的数学环境,而是会有诸多其他方面的要求。这时我们需要更多的数学环境来提供支持。
\end{frame}
\begin{frame}[fragile]{多行公式}{每行均为一个公式}
	\begin{itemize}
		\item 简单的多个公式罗列: \verb|gather(*)|
		\begin{itemize}
			\item 每行公式都是居中显示的
			\item 使用 \cprotect\fbox{\verb|\\|} 换行
			\item 某行不需要编号,在 \cprotect\fbox{\verb|\\|} 前加  \cprotect\fbox{\verb|\notag|} 
		\end{itemize}
		\item 公式按关系符对齐: \verb|align(*)|
		\begin{itemize}
			\item 使用 \cprotect\fbox{\verb|\\|} 换行,关系符前加 \cprotect\fbox{\verb|&|} 表示对齐
			\item \cprotect\fbox{\verb|&|} 一般放在二元关系符($=,<,>$等)前面
			\item 有时可巧用 \cprotect\fbox{\verb|\phantom|} 变通 
		\end{itemize}
		\item 多行公式每行都会产生一个编号,不希望产生的话可以使用\cprotect\fbox{\verb|\notag|}
	\end{itemize}
\end{frame}
\begin{frame}
	练习:尝试输出以下公式。

	\begin{minipage}{0.95\textwidth}
	\begin{gather}
		\partialratio{A}{x}{y} = \lim_{\Delta x \to 0} \frac{f(x+\varDelta x,y)-f(x,y)}{\varDelta x} \\
		\partialratio{A}{y}{x} = \lim_{\Delta y \to 0} \frac{f(x,y+\varDelta y)-f(x,y)}{\varDelta y}
	\end{gather}
	\hrule
	\begin{align}
		& \mathrel{\phantom{=}} (a + b)(a^2 - ab + b^2)\notag \\
		& = a^3 - a^2b + ab^2 + a^2b - ab^2 + b^3\notag \\
		& = a^3 + b^3
	\end{align}
	\end{minipage}
\end{frame}
\begin{frame}[fragile]{多行公式}
	\begin{itemize}
		\item \verb|align(*)| 的另一个用处:排版多列公式
		\begin{itemize}
			\item 每行的基本格式
			\begin{itemize}
				\item[] \verb|a &= b  &  c &= d  &  e &= f \\|
			\end{itemize}
			\item 最后一行不需要 \cprotect\fbox{\verb|\\|}
			\item 按照行来标序
			\item \cprotect\fbox{\verb|&|} 必然为奇数个,出现偶数个会报错(为什么?)
		\end{itemize}
		\item 关于 \verb|flalign(*)|
		\begin{itemize}
			\item 与 \verb|align(*)| 是类似的
			\item 会把多列公式水平方向分散对齐
		\end{itemize}
		\item 关于 \verb|alignat(*)|
		\begin{itemize}
			\item 与 \verb|align(*)| 也是类似的
			\item 不会主动产生间距,需要手动输入间距大小
			\item 需要一个参数表示每行要对齐的公式个数
		\end{itemize}
	\end{itemize}
\end{frame}
\begin{frame}[fragile]
	对比 \verb|align(*)| 与 \verb|flalign(*)|

	\vspace{2ex}
	\clean
	\begin{minipage}{0.49\textwidth}
		\begin{itemize}
			\item 使用 \verb|align|
			\begin{align}
				x + y &= 2 & x &= 1 \\
				x - y &= 0 & y &= 1
			\end{align}
		\end{itemize}
	\end{minipage}
	\clean
	\begin{minipage}{0.49\textwidth}
		\begin{itemize}
			\item 使用 \verb|flalign|
			\begin{flalign}
				x + y &= 2 & x &= 1 \\
				x - y &= 0 & y &= 1
			\end{flalign}
		\end{itemize}
	\end{minipage}

	\vspace{3ex}
	巧用 \verb|alignat|(使用 \verb|align|必须借助 \cprotect\fbox{\verb|\phantom|})

	\begin{minipage}{0.69\textwidth}
\begin{lstlisting}
\begin{alignat*}{5}
    &1 & &+2 & &+3 & &+4 & &=10 \\
    &1 & &   & &+3 & &   & &=4 \\
    &  & &+2 & &   & &+4 & &=6
\end{alignat*}
\end{lstlisting}			
	\end{minipage}
	\begin{minipage}{0.29\textwidth}
		\begin{alignat*}{5}
			&1 & &+2 & &+3 & &+4 & &=10 \\
			&1 & &   & &+3 & &   & &=4 \\
			&  & &+2 & &   & &+4 & &=6
		\end{alignat*}
	\end{minipage}
\end{frame}
\begin{frame}[fragile]{多行公式}
	\begin{itemize}
		\item 如何在多行公式之间插入说明性文字?
		\begin{itemize}
			\item 每次都退出数学环境输入文字过于麻烦
			\item 退出数学环境会破环上下公式间对齐关系
		\end{itemize}
		\item \cprotect\fbox{\verb|\text|} 行内插入文字
		\item \cprotect\fbox{\verb|\intertext|} 换行顶格插入文字(前一行的 \cprotect\fbox{\verb|\\|} 可省略)
		\item \cprotect\fbox{\verb|\shortintertext|} 提供更为紧凑的行间距
	\end{itemize}
\end{frame}
\begin{frame}
	例:

	\vspace{2ex} \clean
	绝热过程中,$Q=0$,不做非膨胀功时,有
	\begin{gather}
		\di U = \delta W
		\quad \text{或} \quad
		\di U + p \di V = 0
		\shortintertext{已知}
		\todi{U}{T}{V}
		\shortintertext{对理想气体,}
		\di U = C_V \di T, \quad \Delta U = \int_{T_1}^{T_2}C_V \di T
		\intertext{若$C_V$为常数,}
		\Delta U = C_V(T_2-T_1) = W
	\end{gather}
\end{frame}
\begin{frame}[fragile]{多行公式}
	\begin{itemize}
		\item 多个联系紧密的公式如何共用一个编号?
		\begin{itemize}
			\item[] 使用 \verb|subequations| 环境
		\end{itemize}
		\item 示例:
\begin{lstlisting}
\begin{subequations}
\begin{align}
  a \cdot b &= b \cdot a \\
  (a \cdot b) \cdot c &= a \cdot (b \cdot c) \\
  a \cdot (b + c) &= a \cdot b + a \cdot c 
\end{align}
\end{subequations}
\end{lstlisting}
		\clean \vspace{-0.5cm}
		\begin{minipage}{0.7\textwidth}
			\begin{subequations}
				\begin{align}
					a \cdot b &= b \cdot a \\
					(a \cdot b) \cdot c &= a \cdot (b \cdot c) \\
					a \cdot (b + c) &= a \cdot b + a \cdot c 
				\end{align}
			\end{subequations}
		\end{minipage}
	\end{itemize}
\end{frame}
\begin{frame}
练习:尝试输出如下的麦克斯韦方程组。

	\vspace{2ex} \clean
\begin{minipage}[t]{0.54\textwidth}
	\begin{itemize}
		\item 积分形式
	\end{itemize}
\end{minipage}
\quad
\begin{minipage}[t]{0.4\textwidth}
	\begin{itemize}
		\item 微分形式
	\end{itemize}
\end{minipage}
\begin{minipage}[c]{0.54\textwidth}
	\begin{subequations}
		\begin{align}
			\varoiint_S \bm{D} \cdot \di \bm{S} &= q, \\
			\varoiint_S \bm{B} \cdot \di \bm{S} &= 0, \\
			\oint_L \bm{E} \cdot \di \bm{\ell} &= - \iint \frac{\partial \bm{B}}{\partial\, t} \cdot \di \bm{S}, \\
			\oint_L \bm{H} \cdot \di \bm{\ell} &= I_0 + \iint \frac{\partial \bm{D}}{\partial\, t} \cdot \di \bm{S}.
		\end{align}
	\end{subequations}
\end{minipage}
\quad
\begin{minipage}[c]{0.4\textwidth}
	\begin{subequations}
		\begin{align}
			\nabla \cdot \bm{D} &= \rho_f, \\
			\nabla \cdot \bm{B} &= 0, \\
			\nabla \times \bm{E} &= -\frac{\partial \bm{B}}{\partial\, t}, \\
			\nabla \times \bm{H} &= \bm{J}_f + \frac{\partial \bm{D}}{\partial\, t}.
		\end{align}
		\end{subequations}		
	\end{minipage}
\end{frame}
\begin{frame}{拆分公式}{整体为一个公式}
	行内公式中,对于长公式如$1+2+3+4+5+6+7+8+9+10+11+12+13+14+15+16+17+18+19+20+21+22+23+24+25+26+27+28+29+30$我们先前提过是可以在二元运算符后换行的,但是对于行间公式则不然,见下例。
	\[
	1+2+3+4+5+6+7+8+9+10+11+12+13+14+15+16+17+18+19+20+21+22+23+24+25+26+27+28+29+30
	\]
	这时候可以使用\texttt{multline(*)}环境。(注意不是 \textcolor{red}{\texttt{multiline}}!)
\end{frame}
\begin{frame}[fragile]
\begin{lstlisting}
\begin{multline}
    1+2+3+4+5+6+7 \\
    +8+9+10+11+12+13 \\
    +14+15+16+17+18+19+20+21 \\
    +22+23+24+25 \\
    +26+27+28+29+30
\end{multline}
\end{lstlisting}
	\clean
	\begin{multline}
		1+2+3+4+5+6+7 \\
		+8+9+10+11+12+13 \\
		+14+15+16+17+18+19+20+21 \\
		+22+23+24+25 \\
		+26+27+28+29+30
	\end{multline}
\end{frame}
\begin{frame}[fragile]{拆分公式}
	\begin{itemize}
		\item 可以看出\texttt{multline}环境中首行是左对齐的,尾行是右对齐的,中间行则是居中
		\item 公式的左右边距是通过长度变量\cprotect\fbox{\verb|\multlinegap|} 和 \cprotect\fbox{\verb|\multlinetaggap|} 设置的
		\item \cprotect\fbox{\verb|\shoveleft{}|} 和 \cprotect\fbox{\verb|\shoveright{}|} 可设置中间行像首尾两行那样左/右对齐
	\end{itemize}
	\setlength{\multlinegap}{4em}
	\setlength{\multlinetaggap}{4em}
	\begin{multline*}
		1+2+3+4+5+6+7 \\
		\shoveright{+8+9+10+11+12+13} \\
		+14+15+16+17+18+19+20+21 \\
		\shoveleft{+22+23+24+25} \\
		+26+27+28+29+30
	\end{multline*}
\end{frame}
\begin{frame}[fragile]
\begin{lstlisting}
\setlength{\multlinegap}{4em}
\setlength{\multlinetaggap}{4em}
\begin{multline*}
    1+2+3+4+5+6+7 \\
    \shoveright{+8+9+10+11+12+13} \\
    +14+15+16+17+18+19+20+21 \\
    \shoveleft{+22+23+24+25} \\
    +26+27+28+29+30
\end{multline*}
\end{lstlisting}
\end{frame}
\begin{frame}[fragile]{拆分公式}
	\begin{itemize}
		\item 如果想要在等号处对齐,可以考虑\texttt{split}环境
		\item \texttt{split}环境应当内嵌于\texttt{equation}中,使用方法上与\texttt{align}等类似
	\end{itemize}
	\clean
	\begin{minipage}{0.43\textwidth}
\begin{lstlisting}
\begin{equation}
\begin{split}
(AB)(B^{-1}A^{-1}) 
&= A(BB^{-1})A^{-1} \\
&= AIA^{-1} 
 = AA^{-1} \\
&= I
\end{split}
\end{equation}
\end{lstlisting}
	\end{minipage}
	\begin{minipage}{0.56\textwidth}
		\begin{equation} \begin{split}
			(AB) (B^{-1}A^{-1}) &= A(BB^{-1})A^{-1} \\
			&= AIA^{-1} = AA^{-1} \\
			&= I
		\end{split} \end{equation}
	\end{minipage}
\end{frame}
\begin{frame}[fragile]{组合成块}
	如何输入分段函数,如
	\[ \sign x = \begin{cases}
	1, & x > 0, \\
	0, & x = 0, \\
	-1, & x < 0.
	\end{cases} \]

	使用\texttt{cases} 环境(同样需要内嵌在\texttt{equation}等环境中)。
\begin{lstlisting}
\[ \sign x = \begin{cases}
  1, & x > 0, \\
  0, & x = 0, \\
  -1, & x < 0.
\end{cases} \]
\end{lstlisting}
\end{frame}

\begin{frame}{组合成块}
	\begin{itemize}
		\item 使用\texttt{dcases}环境得到显示格式大小的公式
		\item 此外使用上述部分环境的\texttt{+ed}模式能得到更多的将公式组合成块的应用
		\begin{itemize}
	 		\item 如:\texttt{gathered}, \texttt{aligned}, \texttt{alignedat}, \texttt{multilined}
	 		\item \texttt{lgathered}与\texttt{rgathered}可以设置环境内的若干行公式左对齐或右对齐,配合定界符使用
	 	\end{itemize}
	 	\item 以上环境的使用方法不变,它们都将多行的公式组合成块放在了一个公式环境内
	 	\item 这里只举几个例子,请自行融会贯通
	\end{itemize}
	\centering
	\texttt{请打开文件equation.tex并编译。}
\end{frame}