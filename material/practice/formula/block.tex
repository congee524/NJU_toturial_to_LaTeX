\documentclass{article}
\usepackage{ctex}
\usepackage{amsmath,mathtools}
\newcommand{\di}{\,\mathrm{d}}
\newcommand{\partialratio}[3]{%
\left(\dfrac{\partial #1}{\partial #2}\right)_{#3}}
\newcommand{\set}[2]{%
\left\{#1 \middle\vert #2 \right\}}
\begin{document}
	由恰当微分的性质可导出Maxwell关系式

	\noindent (按括号对齐)
	\begin{equation}
		\left.
		\begin{rgathered}
			\di U = T \di S - p \di V \\
			\di H = T \di S + V \di p \\
			\di A = - S \di T - p \di V \\
			\di G = - S \di T + V \di p
		\end{rgathered}
		\right\} \Rightarrow \left\{
		\begin{lgathered}
			\partialratio{T}{V}{S} = - \partialratio{p}{S}{V} \\
			\partialratio{T}{p}{S} = \partialratio{V}{S}{p} \\
			\partialratio{S}{V}{T} = \partialratio{p}{T}{V} \\
			\partialratio{S}{p}{T} = - \partialratio{V}{T}{p}
		\end{lgathered}
		\right.
	\end{equation}

	\noindent (按等号对齐)
	\begin{equation}
		\left.
		\begin{aligned}
			\di U &= T \di S - p \di V \\
			\di H &= T \di S + V \di p \\
			\di A &= - S \di T - p \di V \\
			\di G &= - S \di T + V \di p
		\end{aligned}
		\right\} \Rightarrow \left\{
		\begin{aligned}
			\partialratio{T}{V}{S} &= - \partialratio{p}{S}{V} \\
			\partialratio{T}{p}{S} &= \partialratio{V}{S}{p} \\
			\partialratio{S}{V}{T} &= \partialratio{p}{T}{V} \\
			\partialratio{S}{p}{T} &= - \partialratio{V}{T}{p}
		\end{aligned}
		\right.
	\end{equation}

	\vspace{2cm}
	一个高次方程的解集
	\begin{equation}
		\Omega = \set{x}{\begin{multlined}
		x^7 + x^6 + x^5 \\ + x^4 + x^3 + x^2 \\ + x + 1 = 0
		\end{multlined}}
	\end{equation}
\end{document}